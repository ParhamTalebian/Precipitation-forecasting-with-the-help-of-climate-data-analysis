
\documentclass{article}
\usepackage{graphicx}
\usepackage{xepersian}
\settextfont{XB Niloofar}
\title{گزارش پروژه: پیش‌بینی بارش}
\author{پرهام طالبیان و بهزاد اسمی}
\date{\today}

\begin{document}
	
	\maketitle
	
	\section{مقدمه}
	در این پروژه، هدف ما از تحلیل داده و ساخت مدل، پیش‌بینی وقوع بارش بر اساس شرایط آب و هوایی است. بارش یکی از مهم‌ترین وقایع آب و هوایی است که تأثیر زیادی بر زندگی مردم و روند اقتصادی دارد. بررسی دقیق شرایط آب و هوایی و پیش‌بینی بارش می‌تواند به مردم و مدیران آب و هوا در اتخاذ تصمیمات مناسب کمک کند.
	
	\section{تحلیل داده}
	در این مرحله، داده‌های مربوط به شرایط آب و هوایی را از منابع مختلف جمع‌آوری کرده‌ایم، از جمله داده‌های مشاهده شده از ایستگاه‌های هواشناسی. سپس، ابتدا داده‌ها را پیش‌پردازش کرده و سپس تحلیل آماری انجام داده‌ایم. این تحلیل شامل مطالعه توزیع‌های مختلف داده‌ها، بررسی روندها و الگوهای زمانی، و تعیین ویژگی‌های مهم برای مدل‌سازی است.

	

	
	
	

	

	درصد null بودن در هر ویژگی را محاسبه کردیم.
	
	

	و 
		ستون هایی که بیشتر از بیست درصد null دارند را حذف و در ردیف هایی که null دارد نیز حذف کردیم.
		
	


	  دوباره ایندکس بندی کردیم.
	 
	 

	 با استفاده از 
	 \lr{Label encoding}
	 داده های اسمی را به عدد تبدیل کردیم.
	 
	 و 
	 \lr{Atmosphere pressure}
	 را نرمال کردیم تا داده های ما در بازه کمتری باشد.
	 سپس همبستگی بین داده ها را روی هیت مپ پلات کردیم. در قسمت بعد داده های نویز و پرت را حذف کردیم.
	 سپس روباره ایندکس ها را ریست کردیم 
	 
	
	

	
	
		
		
		
		
	
	
	
	\section{ساخت مدل}
	پس از تحلیل داده، سه مدل مختلف برای پیش‌بینی بارش پیاده‌سازی شدند:
	\begin{itemize}
		\item مدل SVM (\lr{Support Vector Machine})
		\item مدل KNN (\lr{K-Nearest Neighbors})
		\item مدل درخت تصمیم
	\end{itemize}
	سپس، با استفاده از داده‌های آموزشی، هر یک از این مدل‌ها آموزش داده شدند و پارامترهای بهینه آن‌ها تنظیم شد.
	
	\section{ارزیابی مدل}
	در این مرحله، عملکرد هر یک از مدل‌ها با استفاده از معیارهای دقت، صحت، و F1-score ارزیابی شد. نتایج ارزیابی برای هر مدل گزارش شده است و مدلی که بهترین عملکرد را داشته است مشخص شده است.
	
	\section{نتیجه‌گیری}
	در این پروژه، با تحلیل دقیق داده‌های مربوط به شرایط آب و هوایی و ساختن مدل‌های پیش‌بینی، به ارزیابی عملکرد مدل‌های مختلف برای پیش‌بینی بارش پرداختیم. نتایج نشان داد که مدل SVM بهترین عملکرد را داشته و می‌تواند به‌طور قابل توجهی دقت پیش‌بینی را افزایش دهد. این پروژه می‌تواند به متخصصان هواشناسی و مدیران در اتخاذ تصمیمات مربوط به مدیریت منابع آب و هوا کمک شایانی کند.
	
\end{document}


